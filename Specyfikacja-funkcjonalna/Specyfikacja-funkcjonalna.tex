\documentclass{report}

\usepackage{polski} % Pozwala na użycie polskiego. Stawia między innymi fontenc na T1
\usepackage[utf8]{inputenc} % Informuje o kodowaniu

\usepackage{xcolor}% http://ctan.org/pkg/xcolor
\usepackage{hyperref}% http://ctan.org/pkg/hyperref

\definecolor{LinkColor}{HTML}{1d5cc1}


\usepackage{tabto}

% Nie twórz nowch stron
\usepackage{etoolbox}
\makeatletter
% \patchcmd{\chapter}{\if@openright\cleardoublepage\else\clearpage\fi}{}{}{}
\makeatother

\title{Specyfikacja funkcjonalna - Gra w życie}
\author{Krzysztof Dąbrowski i Jakub Bogusz}
\date{\today}

\begin{document}
\maketitle{}

\tableofcontents{}

\chapter{Cel projektu}
Celem projektu jest implementacja gry w życie w języku C.

\chapter{Opis ogólny problemu}
Gra w życie jest automatem komórkowym wymyślonym przez brytyjskiego matematyka John Horton Conway %Sprawdzić
w 1970 roku. Polega na symulacji kolejnych pokoleń życia komórek według następujących zasad.

% \subsection{Zasady}

\paragraph{Stany}  Komórka może znajdować się w jednym z dwóch stanów
\begin{itemize}
\item żywa
\item martwa
\end{itemize}

\paragraph{Reguły} Następne pokolenie generowane jest zgodnie z regułami:
\begin{itemize}
\item Jeżeli komóra była martwa i miała dokładnie 3 żywych sąsiadów, w następnym pokoleniu staje się żywa.
\item Jeżeli komóra była żywa to pozostaje żywa jeśli miała dwóch lub trzech żywych sąsiadów. W przeciwnym razie staje się martwa.
\end{itemize}

\chapter{Działanie programu}

\section{Komunikacja z użytkownikiem}

\subsection{Tryb z argumentami z wiersza poleceń}
\paragraph{Argumenty}
\begin{itemize}
\item -h / --help \\ Wyświetlenie pomocy
\item -f [nazwa pliku]  / -{}-file plik=[nazwa pliku] \\ Plik z wejściowym stanem planszy zgodny z \hyperref[format]{\textcolor{LinkColor}{formatem}}. %ToDo: Dać link
% \item -r / -{}-random \\ Jeśli podany, to plansza jest tworzona losowo. Wyklucza się z -f
\item -o [ścieżka] / -{}-output\textunderscore{}dest=[ścieżka]  \\Ścieżka do folderu, w którym zostaną zapisane wyniki symulacji. Domyślnie brak generacji plików i aktywna \hyperref[delay]{\textcolor{LinkColor}{flaga -d 1000}}
\item -t (gif | png) / -{}-type (gif | png)) \\Typ generowanych rezultatów. Domyślnie gif.
\item -n [liczba] / -{}-amount\textunderscore{}of\textunderscore{}generations=[liczba] \\ Ilość pokoleń do wygenerowania. Domyślnie 15
\item -p [liczba] / -{}-step=[liczba] \\ Wybór co który stan symulacji będzie zapisywany. Domyślnie 1
\item -s [liczba] / -{}-size=[liczba] \\ Losowe generowanie planszy początkowej o podanym rozmiarze. Wyklucza się z -f
\item \label{delay} -d [liczba] / -{}-delay=[liczba] \\ Podanie tego argumentu spowoduje wyświetlanie w konsoli kolejnych generacji symulacji. Wartość argumentu [liczba] oznacza czas  w milisekundach między wyświetleniem poszczególnych pokoleń. Domyślnie 1000.
% \item -v / -{}-verboose
\end{itemize}

\subsection{Tryb interaktywny}
Program prowadzi dialog z użytkownikiem pozwalając na wybór wszystkich niezbędnych ustawień.

\section{Format pliku wejściowego}  \label{format}
Przykład: \\
5 3 \tab -- rozmiar (x y) \\
1 0 0 1 1 \tab -- Wartości poszczególnych komórek \\
0 1 1 0 1 \tab -- 1 - żywa \\
0 0 0 1 1 \tab -- 0 - martwa \\

\end{document}