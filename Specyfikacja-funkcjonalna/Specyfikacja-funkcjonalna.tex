\documentclass{report}

\usepackage{polski} % Pozwala na użycie polskiego. Stawia między innymi fontenc na T1
\usepackage[utf8]{inputenc} % Informuje o kodowaniu

\title{Specyfikacja funkcjonalna - Gra w życie}
\author{Krzysztof Dąbrowski i Jakub Bogusz}
\date{\today}

% Nie twórz nowch stron
\usepackage{etoolbox}
\makeatletter
\patchcmd{\chapter}{\if@openright\cleardoublepage\else\clearpage\fi}{}{}{}
\makeatother


\begin{document}
\maketitle{}

\tableofcontents{}

\chapter{Cel projektu}
Celem projektu jest implementacja gry w życie w języku C.

\chapter{Opis ogólny problemu}
Gra w życie jest automatem komórkowym wymyślonym przez brytyjskiego matematyka John Horton Conway %Sprawdzić
w 1970 roku. Polega na symulacji kolejnych pokoleń życia komórek według następujących zasad.

% \subsection{Zasady}

\paragraph{Stany}  Komórka może znajdować się w jednym z dwóch stanów
\begin{itemize}
\item żywa
\item martwa
\end{itemize}

\paragraph{Reguły} Następne pokolenie generowane jest zgodnie z regułami:
\begin{itemize}
\item Jeżeli komóra była martwa i miała dokładnie 3 żywych sąsiadów, w następnym pokoleniu staje się żywa.
\item Jeżeli komóra była żywa to pozostaje żywa jeśli miała dwóch lub trzech żywych sąsiadów. W przeciwnym razie staje się martwa.
\end{itemize}

\end{document}