\documentclass{article}

\usepackage{polski} % Pozwala na użycie polskiego. Ustawia między innymi fontenc na T1
\usepackage[utf8]{inputenc} % Informuje o kodowaniu

\usepackage{xcolor}% http://ctan.org/pkg/xcolor
\usepackage{hyperref}% http://ctan.org/pkg/hyperref

\definecolor{LinkColor}{HTML}{1d5cc1}

\usepackage{tabto}

\usepackage{graphicx} % Pakiet do obrazów
\graphicspath{ {./Obrazy/} } % Folder, z którego będą brane obrazy

% Nie twórz nowych stron
\usepackage{etoolbox}
\makeatletter
% \patchcmd{\chapter}{\if@openright\cleardoublepage\else\clearpage\fi}{}{}{}
\makeatother

\title{Specyfikacja implementacyjna -- Gra w życie}
\author{Krzysztof Dąbrowski i Jakub Bogusz}
\date{\today}

\begin{document}
\maketitle{}

\tableofcontents{}

\section{Podział na moduły}
Program będzie podzielony na współdziałające moduły. Pozwoli to na łatwiejszą modyfikację programu oraz dodawanie nowych funkcjonalności.

\begin{center}
    \textbf{DIAGRAM MODÓŁÓW}
\end{center}

%TODO: Dodać linki
\paragraph{Spis modułów}
\begin{itemize}
    \item GameOfLife
    \item ArgumentsParser
    \item BoardHandler
    \item Simulator
    \item Save
    \item Loader
    \item GraphicsGenerator
\end{itemize}

\subsection{GameOfLife}
Główny moduł kontrolujący przepływ sterowania i danych między pozostałymi modułami.

\subsubsection{Funkcje}
\texttt{main(int argc, char** args)} -- Punkt startowy programu. Z niej wywoływane będą kolejne funkcje.

\end{document}
