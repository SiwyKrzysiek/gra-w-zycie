\documentclass{article}

\usepackage{polski} % Pozwala na użycie polskiego. Ustawia między innymi fontenc na T1
\usepackage[utf8]{inputenc} % Informuje o kodowaniu

\usepackage{xcolor}% http://ctan.org/pkg/xcolor
\usepackage{hyperref}% http://ctan.org/pkg/hyperref

\definecolor{LinkColor}{HTML}{1d5cc1}

\usepackage{tabto}

\usepackage{graphicx} % Pakiet do obrazów
\graphicspath{ {./Obrazy/} } % Folder, z którego będą brane obrazy

% Nie twórz nowych stron
\usepackage{etoolbox}
\makeatletter
% \patchcmd{\chapter}{\if@openright\cleardoublepage\else\clearpage\fi}{}{}{}
\makeatother

\title{Specyfikacja implementacyjna -- Gra w życie}
\author{Krzysztof Dąbrowski i Jakub Bogusz}
\date{\today}

\begin{document}
\maketitle{}

\tableofcontents{}

\section{Podział na moduły}
Program będzie podzielony na współdziałające moduły. Pozwoli to na łatwiejszą modyfikację programu oraz dodawanie nowych funkcjonalności.

\begin{center}
    \textbf{DIAGRAM MODÓŁÓW}
\end{center}

%TODO: Dodać linki
\paragraph{Spis modułów}
\begin{itemize}
    \item GameOfLife
    \item ArgumentsParser
    \item BoardHandler
    \item Simulator
    \item Save
    \item Loader
    \item GraphicsGenerator
\end{itemize}

\subsection{GameOfLife}
Główny moduł kontrolujący przepływ sterowania i danych między pozostałymi modułami.

\subsubsection{Funkcje}
\texttt{int main(int argc, char** args)} -- Punkt startowy programu. Z niej wywoływane będą kolejne funkcje. Przyjmować będzie 2 argumenty -- argumenty wsadowe programu:\\\\
	 \hspace*{10mm}\texttt{int argc} -- ilość argumentów,\\\\
	 \hspace*{10mm}\texttt{char** args} -- tablica napisów -- faktycznych argumentów wywołania	 \hspace*{34mm} programu.\\

\subsection{ArgumentsParser}
Moduł odpowiadający za interpretacje podanych wsadowo argumentów programu, konwersji ich oraz zapisu do utworzonej w tym celu struktury.

\subsubsection{Funkcje}
\texttt{params parseArgs(int argc, char** argv)} -- będzie przetwarzać argumenty wsadowe programu podane w postaci flag na strukturę zawierającą ustawienia symulacji. Przyjmować będzie 2 argumenty -- argumenty wsadowe programu:\\\\
	 \hspace*{10mm}\texttt{int argc} -- ilość argumentów,\\\\
	 \hspace*{10mm}\texttt{char** args} -- tablica napisów, będących faktycznymi argumentami wy-\hspace*{10mm}wołania programu.\\
\\
Zwracać będzie strukturę zawierająca ustawienia symulacji.

\subsubsection{Struktury}
Struktura zawierająca ustawienia symulacji: \\\\
\texttt{typedef struct\{\\
	 \hspace*{10mm}int help;\\
	 \hspace*{10mm}char* file;\\
	 \hspace*{10mm}char* output\_dest;\\
	 \hspace*{10mm}char* type;\\
	 \hspace*{10mm}int number\_of\_generations;\\
	 \hspace*{10mm}int step;\\
	 \hspace*{10mm}int delay;\\
\}params;\\}
\\
\texttt{help} -- informuje o tym czy ma być wyświetlana pomoc,\\
\texttt{file} -- informuje o tym jaka jest ścieżka do pliku wejściowego,\\
\texttt{output\_dest} --  informuje o tym jaka jest ścieżka dla pliku/plików wyjściowych,\\
\texttt{type} -- informuje o tym jaki jest typ zwracanych wyników,\\
\texttt{number\_o\_generations} -- informuje o tym ile pokoleń komórek zostanie wygenerowane,\\
\texttt{step} -- informuje o tym co ile pokoleń zapisywane będą dane wyjściowe,\\
\texttt{delay} -- informuje w jakich odstępach czasu mają się wyświetlać kolejne generacje komórek.\\

\subsection{Simulator}
Moduł odpowiadający za przeprowadzenie właściwej symulacji zgodnie z regułami gry w życie.

\subsubsection{Funkcje}
\texttt{board simulate(board b, params p)} -- będzie przeprowadzać symulacje całej gry w życie:\\\\
	 \hspace*{10mm}\texttt{board b} -- struktura zawierająca stan planszy,\\
	 \hspace*{10mm}\texttt{params p} -- struktura zawierająca ustawienia symulacji.\\
\\
Zwracać będzie strukturę zawierająca końcowy stan planszy.\\\\
\\
\texttt{board nextGen(board b)} -- będzie generować planszę odpowiadającą następnemu pokoleniu komórek. Jako argument będzie przyjmować:\\\\
	 \hspace*{10mm}\texttt{board b} -- struktura zawierająca stan planszy,\\
\\
Zwracać będzie strukturę zawierająca stan planszy w następnym pokoleniu.

\subsection{Loader}
Moduł odpowiedzialny za wczytanie planszy początkowej z pliku tekstowego i zapisanie jej do struktury.

\subsubsection{Funkcje}
\texttt{board load(char* path)} -- będzie przetwarzać plik wejściowy i zapisywać go do struktury. Jako argument będzie przyjmować:\\\\
	 \hspace*{10mm}\texttt{char* path} -- ścieżka do pliku wejściowego,\\
\\
Zwracać będzie strukturę opisującą początkowy stan planszy.\\\\
\\
\texttt{int* getSize(char* path)} -- będzie wczytywać rozmiar planszy zapisanej w pliku wejściowym:\\\\
	 \hspace*{10mm}\texttt{char* path} -- ścieżka do pliku wejściowego,\\
\\
Zwracać będzie dwuelementowy wektor zawierający rozmiary planszy początkowej.
\end{document}
